\documentclass{article}
\usepackage[utf8]{inputenc}
\usepackage{xcolor}
\usepackage{hyperref}
\usepackage{indentfirst}
\usepackage{listings}
\usepackage{indentfirst}

\lstset{
  basicstyle = \small,
  breaklines = true,
  language = bash,
  numbers = left,
  numbersep = 5pt,
  numberstyle = \tiny\color{gray},
  stepnumber = 1,
  commentstyle = \color{gray},
}

%links reference color setup
\hypersetup{
	colorlinks = true,
	urlcolor = blue,
}

%command edits, to delete number of which section but maintaning a number for
%the subsections, also edit the space between line
\makeatletter
\renewcommand\thesection{}
\renewcommand\thesubsection{\@Roman\c@subsection}
\renewcommand{\baselinestretch}{1.2}
\makeatother

%edit in the paragraph length
\setlength{\parindent}{20pt}

\title{Robotic Operating System - ROS}
\author{Salomão Rodrigues Jacinto}

\begin{document}

\maketitle

\section{Quick Guide and Explanation}
We are currently using the Indigo version of ROS which is the last LTS release
and has support for the needed packages until now.

The basic concept of ROS is the message-passing between processes that is made
by topics, for example we have topics publishing informations as odometry and we
can also have topics that are subscribed to the odometry one to receive this
information.

As we go through the packages used, this will be more easy to understand. You
can find in the links below information about ROS, ROS topics and the page where
you can find the tutorials made by ROS.

\begin{itemize}
\item \href{https://en.wikipedia.org/wiki/Robot_Operating_System}{ROS Wikipedia}
\item \href{http://wiki.ros.org/ROS/Tutorials/UnderstandingTopics}{ROS Topics}
\item \href{http://wiki.ros.org/ROS/Tutorials}{ROS Tutorials}
\end{itemize}


\section{Installation}
The complete installation guide can be found at:

\begin{itemize}
\item \href{http://wiki.ros.org/indigo/Installation/Ubuntu}{ROS Installation}
\end{itemize}

As said before we are using Indigo version and to install it you will need
Ubuntu 13.10 or 14.04. Ubuntu is the supported operating system, you can try
install in other distributions but beware you might need to downgrade or install
aditional libraries.
\\
\\
\begin{enumerate}
\item Installation
\\
All repositories need to be allowed, "restricted", "universe" and "multiverse".
\begin{enumerate}

\item Setup sources.list and keys
\begin{lstlisting}
sudo sh -c '. /etc/lsb-release && echo "deb http://ros.fei.edu.br/archive-ros/packages.ros.org/ros/ubuntu $DISTRIB_CODENAME main" > /etc/apt/sources.list.d/ros-latest.list'
sudo apt-key adv --keyserver hkp://ha.pool.sks-keyservers.net --recv-key 0xB01FA116
sudo apt-get update
\end{lstlisting}
\item Install

For full intallation use:
\begin{lstlisting}
sudo apt-get install ros-indigo-desktop-full
\end{lstlisting}

For the robot installation, with no GUI tools use:
\begin{lstlisting}
sudo apt-get install ros-indigo-ros-base
\end{lstlisting}

If you want to search for a packages you can use:
\begin{lstlisting}
sudo apt-cache search ros-indigo-{PAKCAGENAME}
\end{lstlisting}

To easily install system dependencies we need to initialize rosdep:
\begin{lstlisting}
sudo rosdep init
rosdep update
\end{lstlisting}
\end{enumerate}

\item Configuration
\begin{enumerate}

\item To update the ROS environment variables automatically in every new shell:
\begin{lstlisting}
echo "source /opt/ros/indigo/setup.bash" >> ~/.bashrc
source ~/.bashrc
\end{lstlisting}

\item Create a workspace and initialize it:
\begin{lstlisting}
mkdir -p ~/catkin_ws/src
cd ~/catkin_ws/src
catkin_init_workspace
cd ..
catkin_make
echo "source ~/catkin_ws/devel/setup.bash" >> ~/.bashrc
\end{lstlisting}
\end{enumerate}

\item Robot and PC connection
\paragraph{}
Both PC and the robot need to be connected to the same network and some variables
need to be set for them to communicate with each other. Some command lines will
be useful:
\begin{lstlisting}
hostname #to find your pc name
hostname -I #to find your IP address
\end{lstlisting}

\paragraph{}
And you will need to change this two files, "~/.bashrc" and "/etc/hosts".
In the bash file you add the lines below, and in the hosts you add the IP's
followed by their hostnames. The bash and hosts files need to be updated every
time beacuse UFSC has a dynamic IP.
\begin{enumerate}

\item ROS Variables on PC
\begin{lstlisting}
ROS_IP={YOUR-IP}
ROS_HOSTNAME{YOUR-HOSTNAME}
ROS_MASTER_URI=http://{ROBOT-IP}.11311
\end{lstlisting}

\item ROS Variables on the Robot
\begin{lstlisting}
ROS_IP={ROBOT-IP}
ROS_HOSTNAME{ROBOT-HOSTNAME}
ROS_MASTER_URI=http://{ROBOT-IP}.11311
\end{lstlisting}
\end{enumerate}
\end{enumerate}

\section{Packages Used}
\subsection{RosAria}
The RosAria package is used to communicate ROS with ARIA which is the P3-DX
library. It creates topics such as "cmdvel" to publish robot speed and "pose"
to receive robots odometry information.

\begin{enumerate}
\item Installation
\begin{enumerate}

\item First you need to install ARIA, you can download in the link below and
install using "sudo dpkg -i NAME-OF-PACKAGE".
\begin{itemize}
\item \href{http://robots.mobilerobots.com/wiki/ARIA}{ARIA Page}
\end{itemize}

\item Then you go to your source workspace directory and clone the RosAria
repository to it, then you build the package.
\begin{lstlisting}
cd ~/catkin_ws/src
git clone https://github.com/amor-ros-pkg/rosaria.git
catkin_make #if this doesn't work you can try catkin_make --force-cmake
\end{lstlisting}
\end{enumerate}

\item Using RosAria
\begin{enumerate}

\item You need to give write and read permissions to the serial port that the
microcontroller is connected to, and then start the RosAria node.
\begin{lstlisting}
sudo chmod a+rw /dev/ttyS0
rosrun rosaria RosAria _port:=/dev/ttyS0
\end{lstlisting}
\item The laser topic of RosAria doesn't work properly, for that we will use
sicktoolbox.

\end{enumerate}
\end{enumerate}

\subsection{Sicktoolbox Wrapper}

\subsection{Gmapping}

\subsection{Map Server}

\end{document}

\begin{lstlisting}
\end{lstlisting}
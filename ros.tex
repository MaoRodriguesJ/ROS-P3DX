\documentclass{article}
\usepackage[utf8]{inputenc}
\usepackage{xcolor}
\usepackage{hyperref}
\usepackage{indentfirst}
\usepackage{listings}
\usepackage{indentfirst}

\lstset{
  basicstyle = \small,
  breaklines = true,
  language = bash,
  numbers = left,
  numbersep = 5pt,
  numberstyle = \tiny\color{gray},
  stepnumber = 1,
  commentstyle = \color{gray},
}

%links reference color setup
\hypersetup{
	colorlinks = true,
	urlcolor = blue,
}

%command edits, to delete number of which section but maintaning a number for
%the subsections, also edit the space between line
\makeatletter
\renewcommand\thesection{}
\renewcommand\thesubsection{\@Roman\c@subsection}
\renewcommand{\baselinestretch}{1.2}
\makeatother

%edit in the paragraph length
\setlength{\parindent}{20pt}

\title{Robotic Operating System - ROS}
\author{Salomão Rodrigues Jacinto}

\begin{document}

\maketitle

\section{Quick Guide and Explanation}
We are currently using the Indigo version of ROS which is the last LTS release
and has support for the needed packages until now. (ROS Jade has support too).

The basic concept of ROS is the message-passing between processes that is made
by topics, for example we have topics publishing informations as odometry and we
can also have topics that are subscribed to the odometry one to receive this
information.

As we go through the packages used, this will be more easy to understand. You
can find in the links below information about ROS, ROS topics and the page where
you can find the tutorials made by ROS.

You probably can not copy and paste from this document because of the indentation,
so you can follow the links given and copy from there, or can type yourself,
beware of underscores and dashes.

\begin{itemize}
\item \href{https://en.wikipedia.org/wiki/Robot_Operating_System}{ROS Wikipedia}
\item \href{http://wiki.ros.org/ROS/Tutorials/UnderstandingTopics}{ROS Topics}
\item \href{http://wiki.ros.org/ROS/Tutorials}{ROS Tutorials}
\end{itemize}


\section{Installation}
The complete installation guide can be found at:
\begin{itemize}
\item \href{http://wiki.ros.org/indigo/Installation/Ubuntu}{ROS Installation}
\end{itemize}

As said before we are using Indigo version and to install it you will need
Ubuntu 13.10 or 14.04. Ubuntu is the supported operating system, you can try
install in other distributions but beware you might need to downgrade or install
aditional libraries.
\\
\\
\begin{enumerate}

\item Installation
\\
All repositories need to be allowed, "restricted", "universe" and "multiverse".
\begin{enumerate}
\item Setup sources.list and keys
\begin{lstlisting}
sudo sh -c '. /etc/lsb-release && echo "deb http://ros.fei.edu.br/archive-ros/packages.ros.org/ros/ubuntu $DISTRIB_CODENAME main" > /etc/apt/sources.list.d/ros-latest.list'
sudo apt-key adv --keyserver hkp://ha.pool.sks-keyservers.net --recv-key 0xB01FA116
sudo apt-get update

\end{lstlisting}

\item Install

For full intallation use:
\begin{lstlisting}
sudo apt-get install ros-indigo-desktop-full
\end{lstlisting}
For the robot installation, with no GUI tools use:
\begin{lstlisting}
sudo apt-get install ros-indigo-ros-base
\end{lstlisting}
If you want to search for a packages you can use:
\begin{lstlisting}
sudo apt-cache search ros-indigo-{PAKCAGENAME}
\end{lstlisting}
To easily install system dependencies we need to initialize rosdep:
\begin{lstlisting}
sudo rosdep init
rosdep update
\end{lstlisting}
\end{enumerate}

\item Configuration
\begin{enumerate}
\item To update the ROS environment variables automatically in every new shell:
\begin{lstlisting}
echo "source /opt/ros/indigo/setup.bash" >> ~/.bashrc
source ~/.bashrc
\end{lstlisting}
\item Create a workspace and initialize it:
\begin{lstlisting}
mkdir -p ~/catkin_ws/src
cd ~/catkin_ws/src
catkin_init_workspace
cd ..
catkin_make
echo "source ~/catkin_ws/devel/setup.bash" >> ~/.bashrc
\end{lstlisting}
\end{enumerate}

\item Robot and PC connection
\paragraph{}
Both PC and the robot need to be connected to the same network and some variables
need to be set for them to communicate with each other. Some command lines will
be useful:
\begin{lstlisting}
hostname #to find your pc name
hostname -I #to find your IP address
\end{lstlisting}
\paragraph{}
And you will need to change this two files, "~/.bashrc" and "/etc/hosts".
In the bash file you add the lines below, and in the hosts you add the IP's
followed by their hostnames. The bash and hosts files need to be updated every
time beacuse UFSC has a dynamic IP.
\begin{enumerate}
\item ROS Variables on PC
\begin{lstlisting}
ROS_IP={YOUR-IP}
ROS_HOSTNAME{YOUR-HOSTNAME}
ROS_MASTER_URI=http://{ROBOT-IP}.11311
\end{lstlisting}
\item ROS Variables on the Robot
\begin{lstlisting}
ROS_IP={ROBOT-IP}
ROS_HOSTNAME{ROBOT-HOSTNAME}
ROS_MASTER_URI=http://{ROBOT-IP}.11311
\end{lstlisting}
\end{enumerate}
\paragraph{}
The first thing you will run is the main ros node, "roscore", with that running
you can run the nodes from the packages below.
\end{enumerate}

\section{Packages Used}
The RosAria and Sicktoolbox Wrapper need to run on the robot, so you can do that
on the robot itself or through ssh.

\subsection{RosAria}
The RosAria package is used to communicate ROS with ARIA which is the P3-DX
library. It creates topics such as "cmdvel" to publish robot speed and "pose"
to receive robots odometry information.\newline
\begin{enumerate}

\item Installation
\begin{enumerate}
\item First you need to install ARIA, you can download in the link below and
install using "sudo dpkg -i NAME-OF-PACKAGE".
\begin{itemize}
\item \href{http://robots.mobilerobots.com/wiki/ARIA}{ARIA Page}
\end{itemize}
\item Then you go to your source workspace directory and clone the RosAria
repository to it, then you build the package.
\begin{lstlisting}
cd ~/catkin_ws/src
git clone https://github.com/amor-ros-pkg/rosaria.git
cd ~/catkin_ws
catkin_make #if this doesn't work you can try catkin_make --force-cmake
\end{lstlisting}
\end{enumerate}

\item Using RosAria
\begin{enumerate}
\item You need to give write and read permissions to the serial port that the
microcontroller is connected to (in this case 0),
and then start the RosAria node.
\begin{lstlisting}
sudo chmod a+rw /dev/ttyS0
rosrun rosaria RosAria _port:=/dev/ttyS0
\end{lstlisting}
\item The laser topic of RosAria doesn't work properly, for that we will use
sicktoolbox.
\end{enumerate}
\end{enumerate}

\subsection{Sicktoolbox Wrapper}
The Sicktoolbox Wrapper is used to publish laser information in the "scan" topic
properly for further use in the gmapping package.
\begin{enumerate}

\item Installation
\begin{enumerate}
\item First you will need to install some dependencies.
\begin{lstlisting}
sudo apt-get install ros-indigo-sicktoolbox
sudo apt-get install ros-indigo-sicktoolbox-wrapper
\end{lstlisting}
\item Then you can clone the repository into your source directory of ROS
workspace and build the package.
\begin{lstlisting}
cd catkin_ws/src
git clone https://github.com/ros-drivers/sicktoolbox_wrapper
cd catkin_ws
catkin_make
\end{lstlisting}
\end{enumerate}

\item Using Sicktoolbox Wrapper
\begin{enumerate}
\item You need to change the permission of the serial port that the laser is
connected to (in this case 2), and change it's IRQ number.
\begin{lstlisting}
sudo chmod a+rw /dev/ttyS2
sudo setserial /dev/ttyS2 irq 10
\end{lstlisting}
\item Then you can run with the following parameters.
\begin{lstlisting}
rosrun sicktoolbox_wrapper sicklms _port:=/dev/ttyS2 _baud:=38400 _connect_delay:=40
\end{lstlisting}
\item We need this connection delay because the laser becomes available after
turning on after a few seconds.
\end{enumerate}
\end{enumerate}


\subsection{Gmapping}
The Gmapping is used to generate a map from a laser scan.
\begin{enumerate}

\item Installation
\begin{enumerate}
\item First you will need to install some dependencies.
\begin{lstlisting}
sudo apt-get install ros-indigo-gmapping
sudo apt-get install ros-indigo-slam-gmapping
sudo apt-get install ros-indigo-openslam-gmapping
\end{lstlisting}
\item Then you can clone the repository into your source directory and build the
package.
\begin{lstlisting}
cd catkin_ws/src
git clone https://github.com/ros-perception/slam_gmapping
cd catkin_ws
catkin_make
\end{lstlisting}
\end{enumerate}

\item Using Gmapping
\begin{enumerate}
\item Fisrt we need the laser reference frame, the RosAria node already publishes
two frames, the "odom" frame and the "base link" frame, we need a frame called
"laser". The frame transformation below with everything with '0' is just to see
if it's working. We can improve it's precision by doing a right transformation
from the "base link" frame to the "laser one", following this site \href{http://wiki.ros.org/navigation/Tutorials/RobotSetup/TF}{ROS tf Setup}.
\begin{lstlisting}
rosrun tf static_transform_publisher 0 0 0 0 0 0 /base_link /laser /100
\end{lstlisting}
\item Now you can start the Gmapping node and see the mapping on the GUI tool
rviz.
\begin{lstlisting}
rosrun gmapping slam_gmapping scan:=scan
rosrun rviz rviz
\end{lstlisting}
\item In the rviz you can add topics to see what they are publishing in the add
button on the bottom left, and in the right tab of the opening window. To see the
map that is being producing by Gmapping you can add the "/map" topic.
\item Gmapping does not need to be running on the robot, you can start in your PC
if the ROS enviroment variables are correct.
\item To move the robot around to test what it is mapping you can use the
"rosaria client" package, after you clone it's repository you will need to edit
the "print state" file deleting everything referencing to "Bumper State". Then
you can build it corretly and start the teleop node to use the arrow keys to move
around and use the space bar to stop. The repository is: \href{https://github.com/pengtang/rosaria_client}{RosAria Client}
And to run the teleop node:
\begin{lstlisting}
rosrun rosaria_client teleop
\end{lstlisting}
\end{enumerate}
\end{enumerate}

\subsection{Map Server}
\begin{enumerate}

\item Installation
\begin{enumerate}
\end{enumerate}

\item Using Map Server
\begin{enumerate}
\end{enumerate}
\end{enumerate}

\end{document}

\begin{lstlisting}
\end{lstlisting}